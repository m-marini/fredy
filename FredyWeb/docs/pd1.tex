\documentclass{article}
\begin{document}
  \title{Pesi determinanti \thanks{\$Id: pd1.tex,v 1.1 2004/12/28 22:22:01 marco Exp $ $}}
  \author{Marco Marini}
  \maketitle
  
  \part{Definizioni}
  \section{Postulati}
  
  I postulati sono i predicati non dimostrabili.
  Nella rete inferenziale delle regole sono i nodi finali
  
  \section{Evidenze}
  
  Le evidenze sono i predicati dimostrabili.
  Nella rete inferenziale delle regole sono i nodi intermedi.
  
  \section{Ipotesi}
  
  Le ipotesi sono i predicati che non portano a conseguenze.
  Nella rete inferenziale delle regole sono i nodi radice.
  
  \section{Pesi determinanti}
  
  Il calcolo del peso determinante avviene con i seguenti passi:
  
  - Per ogni ipotesi si crea l'insieme dei postulati determinanti per
  l'ipotesi.
  
  - Per ogni postulato si calcola il numero di ipotesi determinante e l'insieme
  unione di tutti i postulati determinanti per ogni ipotesi.

  - Il peso determinante \`e dato dal numero di ipotesi e dalla cardinalit\`a
  dell'insieme unione (numero di postulati condizionatamente determinanti).
  
  Per creare l'insieme dei postulati determinanti per ogni ipotesi si procede
  attraversando il grafo delle regole solo per le espressioni sconosciute,
  quando si arriva ad un'espressione postulato si aggiunge all'insieme dei
  postulati determinanti il postulato attraversato, quando si arriva ad
  un'espressione predicato si ricercano le regole determinanti il predicato.
  
  \section{Ordinamento dei postulati}
  
  L'ordinamento dei postulati avviene mettendo per primi i postulati con grado
  di conoscenza nulla (unknown) e con peso determinante massimo maggiore
  $max(P_0,P_1)$,
  
  a parit\`a di peso determinante massimo maggiore quelli con condizioni
  massime minori $P_0 > P_1 ? C_0 : C_1$ ,
  
  poi i postulati in ordine crescenze di conoscenza (known) e a parid\`a di
  conoscenza quelli con grado di verit� maggiore.
  
\end{document}