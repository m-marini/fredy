\documentclass{article}
\begin{document}
  \title{Pesi determinanti \thanks{\$Id: pd.tex,v 1.3 2004/12/24 14:50:17 marco Exp $ $}}
  \author{Marco Marini}
  \maketitle
  
  \part{Definizioni}
  \section{Postulati}
  
  I postulati sono i predicati non dimostrabili.
  Nella rete inferenziale delle regole sono i nodi finali
  
  \section{Evidenze}
  
  Le evidenze sono i predicati dimostrabili.
  Nella rete inferenziale delle regole sono i nodi intermedi.
  
  \section{Ipotesi}
  
  Le ipotesi sono i predicati che non portano a conseguenze.
  Nella rete inferenziale delle regole sono i nodi radice.

  \section{Pesi determinanti}
  
  Per ogni predicato possiamo definire l'attributo di peso determinante.

  Il peso determinante $P=(p_0,c_0,p_1,c_1)$ \`e un indicatore
  legato alla capacit\`a del  predicato di determinare le ipotesi.
  
  Il peso determinante \`e caratterizzato da quattro valori:
  
  \begin{itemize}
  
  \item
    $p_{0}$ Peso determinante falso
    
    E' il numero massimo di ipotesi determinabili per il valore falso
    
  \item
    $c_{0}$ Condizioni del falso
    
    E' il numero di condizioni per ottenere il numero massimo di ipotesi per il
    valore falso
    
  \item
    $p_{1}$ Peso determinante vero
    
    E' il numero massimo di ipotesi determinabili per il valore vero
    
  \item
    $c_{1}$ Condizioni del vero
    
    E' il numero di condizioni per ottenere il numero massimo di ipotesi per il
    valore vero
    
  \end{itemize}
  
  \part{Funzioni dei nodi del peso determinante}
  
  Determiniamo le funzioni che esprimono come il peso determinante si propaga
  nella rete inferenziale.
  
  \section{Ipotesi}
  
  Il nodo ipotesi \`e il nodo di partenza per l'esplorazione della rete.
  
  Essendo l'ipotesi incondizionatamente determinante solo per se stessa il peso
  determiante dell'ipotesi \`e:
  
  \begin{equation}
      p_0=1, c_0=0, p_1=1, c_1=0
  \end{equation}

  nel caso in cui l'ipotesi sia conosciuta il peso determinante � nullo.
  
  \begin{equation}
      p_0=0, c_0=0, p_1=0, c_1=0
  \end{equation}

  \section{Then-Assign}
  
  Then-Assign sono i nodi che assegnano il valore vero da una regola
  if-then-else per condizione vera.
  
  In questo caso l'evidenza pu\`o solamente essere verificata
  incondizionatamente se l'espressione if \`e vera, quindi se $P_e$ \`e il peso
  determinante dell'evidenza, il peso determinante dell'espressione if \`e:
  
  \begin{equation}
    p_{i0}=0, c_{i0}=0, p_{i1}=p_{e1}, c_{i1}=c_{e1}
  \end{equation}
  
  \section{Then-Not-Assign}
  
  Then-Not-Assign sono i nodi che assegnano il valore false da una regola
  if-then-else per condizione vera.
  
  In questo caso l'evidenza pu\`o solamente essere falsificata
  incondizionatamente se l'espressione if \`e vera, quindi se $P_e$ \`e il peso
  determinante dell'evidenza, il peso determinante dell'espressione if \`e:
  
  \begin{equation}
    p_{i0}=0, c_{i0}=0, p_{i1}=p_{e0}, c_{i1}=c_{e0}
  \end{equation}
  
  \section{Else-Assign}
  
  Else-Assign sono i nodi che assegnano il valore vero da una regola
  if-then-else per condizione falsa.
  
  In questo caso l'evidenza pu\`o solamente essere verificata
  incondizionatamente se l'espressione if \`e false, quindi se $P_e$ \`e il
  peso determinante dell'evidenza, il peso determinante dell'espressione if
  \`e:
  
  \begin{equation}
      p_{i0}=p_{e1}, c_{i0}=c_{e1}, p_{i1}=0, c_{i1}=0
  \end{equation}
  
  \section{Else-Not-Assign}
  
  Else-Not-Assign sono i nodi che assegnano il valore false da una regola
  if-then-else per condizione falsa.
  
  In questo caso l'evidenza pu\`o solamente essere falsificata
  incondizionatamente se l'espressione if \`e false, quindi se $P_e$ \`e il
  peso determinante dell'evidenza, il peso determinante dell'espressione if
  \`e:
  
  \begin{equation}
      p_{i0}=p_{e0}, c_{i0}=c_{e0}, p_{i1}=0, c_{i1}=0
  \end{equation}
  
  \section{Not}
  
  Not sono i nodi che negano un espressione.
  
  Se $P_o$ \`e il peso determinante dell'espressione risultante e $P_i$ il
  peso determinante dell'espressione originale abbiamo:
  
  \begin{equation}
    p_{i0}=p_{o1}, c_{i0}=c_{o0}, p_{i1}=p_{o0}, c_{i1}=c_{o0}
  \end{equation}
  
  \section{Very e Somewhat}
  
  Very e Somewhat sono nodi che cambiano in maniera pi\`u o meno marcata i
  valori
  di un'espressione senza per\`o modificare il senso e quindi non modificano i
  pesi determinanti.
  
  \section{Known}
    
  Known sono i nodi che calcolano il grado di determinanzione di un espressione.
  Il valore risultante  \`e vero se il valore originale \`e vero o falso.
  
  \begin{equation}
    p_{i0}=p_{o1}, c_{i0}=c_{o1}, p_{i1}=p_{o1}, c_{i1}=c_{o1}
  \end{equation}
    
  \section{And}
    
  And sono i nodi che calcolano l'intersezione logica dei valori di ingresso.
  Il valore risultante \`e falso se uno qualsiasi dei valori d'ingressp \`e
  falso.
  
  Il Valore risultate \`e vero se tutti i valori d'ingresso sono veri
  Quindi i pesi determinanti degli ingressi dipendono quindi dai valori
  d'ingresso. In particolare abbiamo due casi:
  
  \begin{enumerate}
    \item Almeno un ingresso falso.
      
      In questo caso il peso determinante \`e nullo
    \item $M$ ingressi indeterminati e $N-M$ ingressi veri
      
      In questo caso il peso determinante \`e:
      \begin{equation}
	p_{i0}=p_{e0}, c_{i0}=c_{e0}, p_{i1}=p_{e1}, c_{i1}=c_{e1}+M-1
      \end{equation}
    \item
  \end{enumerate}
  
  \section{Or}
  
  Or sono i nodi che calcolano l'unione logica dei valori di ingresso.
  Il valore risultante \`e vero se uno qualsiasi dei valori d'ingresso \`e
  vero.
  
  Il Valore risultate \`e falso se tutti i valori d'ingresso sono falsi
  Quindi i pesi determinanti degli ingressi dipendono quindi dai valori
  d'ingresso. In particolare abbiamo due casi:
  
  \begin{enumerate}
    \item Almeno un ingresso vero.
      
      In questo caso il peso determinante \`e nullo
    \item $M$ ingressi indeterminati e $N-M$ ingressi falsi
      
      In questo caso il peso determinante \`e:
      \begin{equation}
	p_{i0}=p_{e0}, c_{i0}=c_{e0}+M-1, p_{i1}=p_{e1}, c_{i1}=c_{e1}
      \end{equation}
  \end{enumerate}
  
  \section{IfOnlyIf}
    
  IfOnlyIf sono i nodi che sono veri se ambedue gli ingressi hanno lo stesso
  valore.
  
  La tabella della verit\`a \`e:
  
\[
    \begin{array}{llll}
      & A& 0 & 1
      \\
      B& 0 & 1 & 0
      \\
      & 1 & 0 & 1
    \end{array}
\]
 
  Abbiamo allora due casi:
  \begin{enumerate}
    \item l'altro ingresso \`e falso
 
      Il peso determinante \`e:
      \begin{equation}
	p_{i0}=p_{e1}, c_{i0}=c_{e1}, p_{i1}=p_{e0}, c_{i1}=c_{e0}
      \end{equation}

    \item l'altro ingresso \`e vero
 
      Il peso determinante \`e:
      \begin{equation}
	p_{i0}=p_{e0}, c_{i0}=c_{e0}, p_{i1}=p_{e1}, c_{i1}=c_{e1}
      \end{equation}

    \item l'altro ingresso \`e sconosciuto
 
      Il peso determinante \`e:
      \begin{equation}
	  p_{i0}=min(p_{e0},p_{e1}), c_{i0}=(p_{e0}< p_{e1} ? c_{e0} : c_{e1})+1,
	  p_{i1}=p_{i0}, c_{i1}=c_{i0}
      \end{equation}

  \end{enumerate}
  
  \section{Implies}
  
  Implies sono i nodi che sono falsi se il primo parametro \`e vero e il
  secondo \`e falso.
  La tabella della verit\`a \`e:
  
  \[
    \begin{array}{llll}
      & A& 0 & 1
      \\
      B& 0 & 1 & 0
      \\
      & 1 & 1 & 1
    \end{array}
  \]
  
  Per il primo parametero abbiamo allora i tre casi:
  \begin{enumerate}
    \item il secondo parametro \`e falso
 
      Il peso determinante del primo parametro \`e:
      \begin{equation}
	p_{i0}=p_{e1}, c_{i0}=c_{e1}, p_{i1}=p_{e0},c_{i1}=c_{e0}
      \end{equation}

    \item il secondo parametro \`e vero
      
      Il peso determinante del primo parametro \`e nullo:
      
    \item il secondo parametro \`e sconosciuto
 
      Il peso determinante del primo parametro \`e:
      
      \begin{equation}
	p_{i0}=min(p_{e0},p_{e1}), c_{i0}=p_{e0}< p_{e1} ? c_{e0}+1 : c_{e1} + 1,
	  p_{i1}=p_{i0}, c_{i1}=c_{i0}
      \end{equation}

  \end{enumerate}
  
  Per il secondo parametero abbiamo allora i tre casi:
  \begin{enumerate}
    \item il primo parametro \`e falso
 
      Il peso determinante del secondo parametro \`e nullo.

    \item il primo parametro \`e vero
      
      \begin{equation}
	p_{i0}=p_{e0}, c_{i0}=c_{e0}, p_{i1}=p_{e1}, c_{i1}=c_{e1}
      \end{equation}

    \item il primo parametro \`e sconosciuto
 
      Il peso determinante del secondo parametro \`e:      
      \begin{equation}
	p_{i0}=min(p_{e0},p_{e1}), c_{i0}=p_{e0}< p_{e1} ? c_{e0} : c_{e1} + 1,
	p_{i1}=p_{e1}, c_{i1}=c_{e1}
      \end{equation}

  \end{enumerate}

  \section{Ordinamento dei postulati}
  
  L'ordinamento dei postulati avviene mettendo per primi i postulati con grado
  di conoscenza nulla (unknown) e con peso determinante massimo maggiore
  $max(P_0,P_1)$,
  
  a parit\`a di peso determinante massimo maggiore quelli con condizioni
  massime minori $P_0 > P_1 ? C_0 : C_1$ ,
  
  poi i postulati in ordine crescenze di conoscenza (known) e a parid\`a di
  conoscenza quelli con grado di verit� maggiore.
  
\end{document}